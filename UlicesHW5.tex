% Options for packages loaded elsewhere
\PassOptionsToPackage{unicode}{hyperref}
\PassOptionsToPackage{hyphens}{url}
%
\documentclass[
]{article}
\usepackage{amsmath,amssymb}
\usepackage{iftex}
\ifPDFTeX
  \usepackage[T1]{fontenc}
  \usepackage[utf8]{inputenc}
  \usepackage{textcomp} % provide euro and other symbols
\else % if luatex or xetex
  \usepackage{unicode-math} % this also loads fontspec
  \defaultfontfeatures{Scale=MatchLowercase}
  \defaultfontfeatures[\rmfamily]{Ligatures=TeX,Scale=1}
\fi
\usepackage{lmodern}
\ifPDFTeX\else
  % xetex/luatex font selection
\fi
% Use upquote if available, for straight quotes in verbatim environments
\IfFileExists{upquote.sty}{\usepackage{upquote}}{}
\IfFileExists{microtype.sty}{% use microtype if available
  \usepackage[]{microtype}
  \UseMicrotypeSet[protrusion]{basicmath} % disable protrusion for tt fonts
}{}
\makeatletter
\@ifundefined{KOMAClassName}{% if non-KOMA class
  \IfFileExists{parskip.sty}{%
    \usepackage{parskip}
  }{% else
    \setlength{\parindent}{0pt}
    \setlength{\parskip}{6pt plus 2pt minus 1pt}}
}{% if KOMA class
  \KOMAoptions{parskip=half}}
\makeatother
\usepackage{xcolor}
\usepackage[margin=1in]{geometry}
\usepackage{color}
\usepackage{fancyvrb}
\newcommand{\VerbBar}{|}
\newcommand{\VERB}{\Verb[commandchars=\\\{\}]}
\DefineVerbatimEnvironment{Highlighting}{Verbatim}{commandchars=\\\{\}}
% Add ',fontsize=\small' for more characters per line
\usepackage{framed}
\definecolor{shadecolor}{RGB}{248,248,248}
\newenvironment{Shaded}{\begin{snugshade}}{\end{snugshade}}
\newcommand{\AlertTok}[1]{\textcolor[rgb]{0.94,0.16,0.16}{#1}}
\newcommand{\AnnotationTok}[1]{\textcolor[rgb]{0.56,0.35,0.01}{\textbf{\textit{#1}}}}
\newcommand{\AttributeTok}[1]{\textcolor[rgb]{0.13,0.29,0.53}{#1}}
\newcommand{\BaseNTok}[1]{\textcolor[rgb]{0.00,0.00,0.81}{#1}}
\newcommand{\BuiltInTok}[1]{#1}
\newcommand{\CharTok}[1]{\textcolor[rgb]{0.31,0.60,0.02}{#1}}
\newcommand{\CommentTok}[1]{\textcolor[rgb]{0.56,0.35,0.01}{\textit{#1}}}
\newcommand{\CommentVarTok}[1]{\textcolor[rgb]{0.56,0.35,0.01}{\textbf{\textit{#1}}}}
\newcommand{\ConstantTok}[1]{\textcolor[rgb]{0.56,0.35,0.01}{#1}}
\newcommand{\ControlFlowTok}[1]{\textcolor[rgb]{0.13,0.29,0.53}{\textbf{#1}}}
\newcommand{\DataTypeTok}[1]{\textcolor[rgb]{0.13,0.29,0.53}{#1}}
\newcommand{\DecValTok}[1]{\textcolor[rgb]{0.00,0.00,0.81}{#1}}
\newcommand{\DocumentationTok}[1]{\textcolor[rgb]{0.56,0.35,0.01}{\textbf{\textit{#1}}}}
\newcommand{\ErrorTok}[1]{\textcolor[rgb]{0.64,0.00,0.00}{\textbf{#1}}}
\newcommand{\ExtensionTok}[1]{#1}
\newcommand{\FloatTok}[1]{\textcolor[rgb]{0.00,0.00,0.81}{#1}}
\newcommand{\FunctionTok}[1]{\textcolor[rgb]{0.13,0.29,0.53}{\textbf{#1}}}
\newcommand{\ImportTok}[1]{#1}
\newcommand{\InformationTok}[1]{\textcolor[rgb]{0.56,0.35,0.01}{\textbf{\textit{#1}}}}
\newcommand{\KeywordTok}[1]{\textcolor[rgb]{0.13,0.29,0.53}{\textbf{#1}}}
\newcommand{\NormalTok}[1]{#1}
\newcommand{\OperatorTok}[1]{\textcolor[rgb]{0.81,0.36,0.00}{\textbf{#1}}}
\newcommand{\OtherTok}[1]{\textcolor[rgb]{0.56,0.35,0.01}{#1}}
\newcommand{\PreprocessorTok}[1]{\textcolor[rgb]{0.56,0.35,0.01}{\textit{#1}}}
\newcommand{\RegionMarkerTok}[1]{#1}
\newcommand{\SpecialCharTok}[1]{\textcolor[rgb]{0.81,0.36,0.00}{\textbf{#1}}}
\newcommand{\SpecialStringTok}[1]{\textcolor[rgb]{0.31,0.60,0.02}{#1}}
\newcommand{\StringTok}[1]{\textcolor[rgb]{0.31,0.60,0.02}{#1}}
\newcommand{\VariableTok}[1]{\textcolor[rgb]{0.00,0.00,0.00}{#1}}
\newcommand{\VerbatimStringTok}[1]{\textcolor[rgb]{0.31,0.60,0.02}{#1}}
\newcommand{\WarningTok}[1]{\textcolor[rgb]{0.56,0.35,0.01}{\textbf{\textit{#1}}}}
\usepackage{graphicx}
\makeatletter
\def\maxwidth{\ifdim\Gin@nat@width>\linewidth\linewidth\else\Gin@nat@width\fi}
\def\maxheight{\ifdim\Gin@nat@height>\textheight\textheight\else\Gin@nat@height\fi}
\makeatother
% Scale images if necessary, so that they will not overflow the page
% margins by default, and it is still possible to overwrite the defaults
% using explicit options in \includegraphics[width, height, ...]{}
\setkeys{Gin}{width=\maxwidth,height=\maxheight,keepaspectratio}
% Set default figure placement to htbp
\makeatletter
\def\fps@figure{htbp}
\makeatother
\setlength{\emergencystretch}{3em} % prevent overfull lines
\providecommand{\tightlist}{%
  \setlength{\itemsep}{0pt}\setlength{\parskip}{0pt}}
\setcounter{secnumdepth}{-\maxdimen} % remove section numbering
\ifLuaTeX
  \usepackage{selnolig}  % disable illegal ligatures
\fi
\usepackage{bookmark}
\IfFileExists{xurl.sty}{\usepackage{xurl}}{} % add URL line breaks if available
\urlstyle{same}
\hypersetup{
  pdftitle={Chapter 3 HW},
  pdfauthor={Ulices Gonzalez},
  hidelinks,
  pdfcreator={LaTeX via pandoc}}

\title{Chapter 3 HW}
\author{Ulices Gonzalez}
\date{March 6th, 2025}

\begin{document}
\maketitle

\begin{center}\rule{0.5\linewidth}{0.5pt}\end{center}

\subsection{Conceptual Questions}\label{conceptual-questions}

\subsubsection{3. We now review k-fold
cross-validation.}\label{we-now-review-k-fold-cross-validation.}

\paragraph{(a) Explain how k-fold cross-validation is
implemented.}\label{a-explain-how-k-fold-cross-validation-is-implemented.}

Set of observations are divided in to k groups of equal size. Each
group/fold is treated as a validation set and the method is fit on the
remaining folds. The MSE is used then found using the observations in
the held out groups/folds. This is repeated k times with a different
group being used as a validation each time. At the end averaging the k
test errors will result in the k-fold cv estimate.

\paragraph{(b) What are the advantages and disadvantages of k-fold cross
validation relative
to:}\label{b-what-are-the-advantages-and-disadvantages-of-k-fold-cross-validation-relative-to}

\subparagraph{i. The validation set
approach?}\label{i.-the-validation-set-approach}

\begin{itemize}
\tightlist
\item
  K-Fold gives data on the variability of the models performance through
  the use of using k amount of error estimates.
\item
  There is less bias in the model since it is trained and validation
  using all of the data its provided in all of its iterations.
\item
  Due to each iteration using \(\frac{k-1}{k}\) of the data set its more
  efficient in how it handles data.
\end{itemize}

\subparagraph{ii. LOOCV?}\label{ii.-loocv}

\begin{itemize}
\tightlist
\item
  Given that there is less bias present in the k-fold method this also
  means that it has higher variance than normal even with a small k.
\item
  The use of k requires the model to be fit k times which can be
  inefficient especially if the data set being used is large.
\end{itemize}

\begin{center}\rule{0.5\linewidth}{0.5pt}\end{center}

\subsection{Applied Questions}\label{applied-questions}

\subsubsection{5. In Chapter 4, we used logistic regression to predict
the probability of default using income and balance on the Default data
set. We will now estimate the test error of this logistic regression
model using the validation set approach. Do not forget to set a random
seed before beginning your
analysis.}\label{in-chapter-4-we-used-logistic-regression-to-predict-the-probability-of-default-using-income-and-balance-on-the-default-data-set.-we-will-now-estimate-the-test-error-of-this-logistic-regression-model-using-the-validation-set-approach.-do-not-forget-to-set-a-random-seed-before-beginning-your-analysis.}

\begin{Shaded}
\begin{Highlighting}[]
\NormalTok{default }\OtherTok{\textless{}{-}} \FunctionTok{read.csv}\NormalTok{(}\StringTok{"DataSets/Default.csv"}\NormalTok{)}
\NormalTok{default}\SpecialCharTok{$}\NormalTok{default }\OtherTok{\textless{}{-}} \FunctionTok{ifelse}\NormalTok{(default}\SpecialCharTok{$}\NormalTok{default }\SpecialCharTok{==} \StringTok{"Yes"}\NormalTok{, }\DecValTok{1}\NormalTok{, }\DecValTok{0}\NormalTok{)}
\NormalTok{default}\SpecialCharTok{$}\NormalTok{student }\OtherTok{\textless{}{-}} \FunctionTok{ifelse}\NormalTok{(default}\SpecialCharTok{$}\NormalTok{student }\SpecialCharTok{==} \StringTok{"Yes"}\NormalTok{, }\DecValTok{1}\NormalTok{, }\DecValTok{0}\NormalTok{)}
\FunctionTok{set.seed}\NormalTok{(}\DecValTok{1}\NormalTok{)}
\end{Highlighting}
\end{Shaded}

\paragraph{(a) Fit a logistic regression model that uses income and
balance to predict
default.}\label{a-fit-a-logistic-regression-model-that-uses-income-and-balance-to-predict-default.}

\begin{Shaded}
\begin{Highlighting}[]
\NormalTok{defaultModel }\OtherTok{\textless{}{-}} \FunctionTok{glm}\NormalTok{(default }\SpecialCharTok{\textasciitilde{}}\NormalTok{ income }\SpecialCharTok{+}\NormalTok{ balance, }\AttributeTok{data =}\NormalTok{ default, }\AttributeTok{family =}\NormalTok{ binomial)}
\FunctionTok{summary}\NormalTok{(defaultModel)}
\end{Highlighting}
\end{Shaded}

\begin{verbatim}
## 
## Call:
## glm(formula = default ~ income + balance, family = binomial, 
##     data = default)
## 
## Coefficients:
##               Estimate Std. Error z value Pr(>|z|)    
## (Intercept) -1.154e+01  4.348e-01 -26.545  < 2e-16 ***
## income       2.081e-05  4.985e-06   4.174 2.99e-05 ***
## balance      5.647e-03  2.274e-04  24.836  < 2e-16 ***
## ---
## Signif. codes:  0 '***' 0.001 '**' 0.01 '*' 0.05 '.' 0.1 ' ' 1
## 
## (Dispersion parameter for binomial family taken to be 1)
## 
##     Null deviance: 2920.6  on 9999  degrees of freedom
## Residual deviance: 1579.0  on 9997  degrees of freedom
## AIC: 1585
## 
## Number of Fisher Scoring iterations: 8
\end{verbatim}

\paragraph{(b) Using the validation set approach, estimate the test
error of this model. In order to do this, you must perform the following
steps:}\label{b-using-the-validation-set-approach-estimate-the-test-error-of-this-model.-in-order-to-do-this-you-must-perform-the-following-steps}

\subparagraph{i. Split the sample set into a training set and a
validation
set.}\label{i.-split-the-sample-set-into-a-training-set-and-a-validation-set.}

\subparagraph{ii. Fit a multiple logistic regression model using only
the training
observations.}\label{ii.-fit-a-multiple-logistic-regression-model-using-only-the-training-observations.}

\subparagraph{iii. Obtain a prediction of default status for each
individual in the validation set by computing the posterior probability
of default for that individual, and classifying the individual to the
default category if the posterior probability is greater than
0.5.}\label{iii.-obtain-a-prediction-of-default-status-for-each-individual-in-the-validation-set-by-computing-the-posterior-probability-of-default-for-that-individual-and-classifying-the-individual-to-the-default-category-if-the-posterior-probability-is-greater-than-0.5.}

\subparagraph{iv. Compute the validation set error, which is the
fraction of the observations in the validation set that are
misclassifed.}\label{iv.-compute-the-validation-set-error-which-is-the-fraction-of-the-observations-in-the-validation-set-that-are-misclassifed.}

\paragraph{(c) Repeat the process in (b) three times, using three
different splits of the observations into a training set and a
validation set. Comment on the results
obtained.}\label{c-repeat-the-process-in-b-three-times-using-three-different-splits-of-the-observations-into-a-training-set-and-a-validation-set.-comment-on-the-results-obtained.}

\paragraph{(d) Now consider a logistic regression model that predicts
the probability of default using income, balance, and a dummy variable
for student. Estimate the test error for this model using the validation
set approach. Comment on whether or not including a dummy variable for
student leads to a reduction in the test error
rate.}\label{d-now-consider-a-logistic-regression-model-that-predicts-the-probability-of-default-using-income-balance-and-a-dummy-variable-for-student.-estimate-the-test-error-for-this-model-using-the-validation-set-approach.-comment-on-whether-or-not-including-a-dummy-variable-for-student-leads-to-a-reduction-in-the-test-error-rate.}

\subsubsection{6. We continue to consider the use of a logistic
regression model to predict the probability of default using income and
balance on the Default data set. In particular, we will now compute
estimates for the standard errors of the income and balance logistic
regression coefficients in two different ways: (1) using the bootstrap,
and (2) using the standard formula for computing the standard errors in
the glm() function. Do not forget to set a random seed before beginning
your
analysis.}\label{we-continue-to-consider-the-use-of-a-logistic-regression-model-to-predict-the-probability-of-default-using-income-and-balance-on-the-default-data-set.-in-particular-we-will-now-compute-estimates-for-the-standard-errors-of-the-income-and-balance-logistic-regression-coefficients-in-two-different-ways-1-using-the-bootstrap-and-2-using-the-standard-formula-for-computing-the-standard-errors-in-the-glm-function.-do-not-forget-to-set-a-random-seed-before-beginning-your-analysis.}

\paragraph{(a) Using the summary() and glm() functions, determine the
estimated standard errors for the coefficients associated with income
and balance in a multiple logistic regression model that uses both
predictors.}\label{a-using-the-summary-and-glm-functions-determine-the-estimated-standard-errors-for-the-coefficients-associated-with-income-and-balance-in-a-multiple-logistic-regression-model-that-uses-both-predictors.}

\paragraph{(b) Write a function, boot.fn(), that takes as input the
Default data set as well as an index of the observations, and that
outputs the coefficient estimates for income and balance in the multiple
logistic regression
model.}\label{b-write-a-function-boot.fn-that-takes-as-input-the-default-data-set-as-well-as-an-index-of-the-observations-and-that-outputs-the-coefficient-estimates-for-income-and-balance-in-the-multiple-logistic-regression-model.}

\paragraph{(c) Use the boot() function together with your boot.fn()
function to estimate the standard errors of the logistic regression
coefficients for income and
balance.}\label{c-use-the-boot-function-together-with-your-boot.fn-function-to-estimate-the-standard-errors-of-the-logistic-regression-coefficients-for-income-and-balance.}

\paragraph{(d) Comment on the estimated standard errors obtained using
the glm() function and using your bootstrap
function.}\label{d-comment-on-the-estimated-standard-errors-obtained-using-the-glm-function-and-using-your-bootstrap-function.}

\subsubsection{7. In Sections 5.3.2 and 5.3.3, we saw that the cv.glm()
function can be used in order to compute the LOOCV test error estimate.
Alternatively, one could compute those quantities using just the glm()
and predict.glm() functions, and a for loop. You will now take this
approach in order to compute the LOOCV error for a simple logistic
regression model on the Weekly data set. Recall that in the context of
classification problems, the LOOCV error is given in
(5.4).}\label{in-sections-5.3.2-and-5.3.3-we-saw-that-the-cv.glm-function-can-be-used-in-order-to-compute-the-loocv-test-error-estimate.-alternatively-one-could-compute-those-quantities-using-just-the-glm-and-predict.glm-functions-and-a-for-loop.-you-will-now-take-this-approach-in-order-to-compute-the-loocv-error-for-a-simple-logistic-regression-model-on-the-weekly-data-set.-recall-that-in-the-context-of-classification-problems-the-loocv-error-is-given-in-5.4.}

\paragraph{(a) Fit a logistic regression model that predicts Direction
using Lag1 and
Lag2.}\label{a-fit-a-logistic-regression-model-that-predicts-direction-using-lag1-and-lag2.}

\paragraph{(b) Fit a logistic regression model that predicts Direction
using Lag1 and Lag2 using all but the first
observation.}\label{b-fit-a-logistic-regression-model-that-predicts-direction-using-lag1-and-lag2-using-all-but-the-first-observation.}

\paragraph{(c) Use the model from (b) to predict the direction of the
first observation. You can do this by predicting that the first
observation will go up if P(Direction = ``Up''\textbar Lag1, Lag2)
\textgreater{} 0.5. Was this observation correctly
classified?}\label{c-use-the-model-from-b-to-predict-the-direction-of-the-first-observation.-you-can-do-this-by-predicting-that-the-first-observation-will-go-up-if-pdirection-uplag1-lag2-0.5.-was-this-observation-correctly-classified}

\paragraph{(d) Write a for loop from i = 1 to i = n, where n is the
number of observations in the data set, that performs each of the
following
steps:}\label{d-write-a-for-loop-from-i-1-to-i-n-where-n-is-the-number-of-observations-in-the-data-set-that-performs-each-of-the-following-steps}

\subparagraph{i. Fit a logistic regression model using all but the ith
observation to predict Direction using Lag1 and
Lag2.}\label{i.-fit-a-logistic-regression-model-using-all-but-the-ith-observation-to-predict-direction-using-lag1-and-lag2.}

\subparagraph{ii. Compute the posterior probability of the market moving
up for the ith
observation.}\label{ii.-compute-the-posterior-probability-of-the-market-moving-up-for-the-ith-observation.}

\subparagraph{iii. Use the posterior probability for the ith observation
in order to predict whether or not the market moves
up.}\label{iii.-use-the-posterior-probability-for-the-ith-observation-in-order-to-predict-whether-or-not-the-market-moves-up.}

\subparagraph{iv. Determine whether or not an error was made in
predicting the direction for the ith observation. If an error was
made,then indicate this as a 1, and indicate it as a
0.}\label{iv.-determine-whether-or-not-an-error-was-made-in-predicting-the-direction-for-the-ith-observation.-if-an-error-was-madethen-indicate-this-as-a-1-and-indicate-it-as-a-0.}

\paragraph{(e) Take the average of the n numbers obtained in (d)iv in
order to obtain the LOOCV estimate for the test error. Comment on the
results.}\label{e-take-the-average-of-the-n-numbers-obtained-in-div-in-order-to-obtain-the-loocv-estimate-for-the-test-error.-comment-on-the-results.}

\end{document}
